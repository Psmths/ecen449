\documentclass[border=5pt, 10pt]{article}
\usepackage{amsmath,amssymb,listings,fancyvrb,rotating,graphicx,hyperref}
\usepackage[a4paper,margin=0.5in,footskip=0.25in]{geometry}
\setcounter{secnumdepth}{0}
\lstset{numbersep=5pt,xleftmargin=0.10in, xrightmargin=.15in} 
\usepackage[activate={true,nocompatibility},final,tracking=true,kerning=true,spacing=true,factor=1200,stretch=10,shrink=10]{microtype}
\usepackage{lscape}
\graphicspath{ {.} }
\title{ECEN 449 Lab Report 4}
\date{\today}
\author{Philip Smith - 624002014 (Sec. 511)}

\begin{document}

    \clearpage\maketitle
    \thispagestyle{empty}
    \newpage
    \setcounter{page}{1}

    {\noindent\Huge Introduction\\\\}
        \noindent Similar to the previous lab this time we are tasked with synthesizing and implementing a Zynq processor. Instead of only running one program on it, however, we will compile and install the Linux kernel onto it. This is accomplished by installing a series of boot loaders that will eventually lead to a normal Linux boot procedure from an initial ramdisk. This particular computer will have DDR3 RAM, a memory management unit, SD card peripheral, and interrup controller, as well as a timer.\\

    \newpage
    
    {\noindent\Huge Procedure\\\\}
        \noindent We follow roughly the same procedure as in Lab 3 by creating the Zynq 7 processor, however this time we will enable several additional peripherals to be used by the kernel in addition to our multiply IP module.\\
        
        \noindent In addition to burning the Zynq 7 on the FPGA we must also compile U-boot and the Linux kernel. \\

    \newpage
    
    {\noindent\Huge Results\\\\}
        \noindent 
    \newpage
    
    {\noindent\Huge Conclusions\\\\}
        \noindent  
    \newpage
    
    {\noindent\Huge Questions\\\\}
        \begin{enumerate}
            \item{}
            \item{}
            \item{}
        \end{enumerate}
    \newpage
    
    \begin{landscape}
    {\noindent\Huge Appendix A - \\\\}
        \noindent The following is the code for :
        %\VerbatimInput{./lab03.sdk/multiply_test/src/helloworld.c}
    \end{landscape}
    \newpage
    
\end{document}
